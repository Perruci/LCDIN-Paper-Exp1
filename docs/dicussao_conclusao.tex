\section{Discussão}

\subsection{Estimação de Parâmetros}
A partir do sinal de posição do carro filtrado, expresso na figura \ref{fig::step_response}, foi possível estimar os parâmetros: tempo de pico ($t_p$), valor de pico ($V_p$) e comportamento em regime permanente do sistema ($G_{ss}$), como do sinal de referência ($R_{ss}$). Para tanto, utilizou-se a função \textit{stepinfo()} da toolbox de sistemas de controle do MATLAB \cite{Grace:1992:MATLAB_Control_Toolbox}. Esses valores encontram-se expressos graficamente na figura \ref{fig::step_analysis} e na equação \ref{eq::dis::step_analysis}. O tempo de pico mostrado é o relativo entre o pico da figura \ref{fig::step_analysis} e o último instante próximo de zero: $t_p = 1.538 - 1.463$ segundos.

\begin{equation}
\begin{aligned}
& t_p = 0.075s & V_p = 1.1565V \\
& R_{ss} = 0.98V & G_{ss} = 0.95V \\
\end{aligned}
\label{eq::dis::step_analysis}
\end{equation}

Note que, a partir dos valores estimados na equação \ref{eq::dis::step_analysis} e da equação \ref{eq-tempo-pico}, é possível estimar a frequência natural do sistema $\omega_n$. De forma semelhante, é possível combinar as equações \ref{eq-zeta}, \ref{eq-Sobressinal} e \ref{eq::dis::step_analysis} de forma a computar o coeficiente de amortecimento $\zeta$. Os valores correspondentes a $\omega_n$ e $\zeta $ encontram se expressos nas equações \ref{eq::dis::natural_frequency} e \ref{eq::dis::zeta}, respectivamente.

\begin{equation}
\omega_n = \frac{\pi}{t_p} = 41.8879\nicefrac{rad}{s}
\label{eq::dis::natural_frequency}
\end{equation}

\begin{equation}
\zeta = 0.4313
\label{eq::dis::zeta}
\end{equation}

Torna-se possível, portanto, estimar os parâmetros $K_m$ e $T_m$. Podemos reescrever as equações expressas em \ref{eq::comparison-second-order-plant}, conforme as equações \ref{eq::dis::km} e \ref{eq::dis::tm}. Substituindo-se as expressões de $k$, $k_p$, $\omega_n$ e $\zeta$ das equações \ref{eq::ganho_nao_inversor}, \ref{eq::ganho_pot}, \ref{eq::dis::natural_frequency} e \ref{eq::dis::zeta}, chega-se a valores numéricos para $k_m$ e $T_m$.

\begin{equation}
K_m = \frac{\omega_n}{2 \zeta k_p k } = 8.55 
\label{eq::dis::km}
\end{equation}

\begin{equation}
T_m = \frac{1}{2 \zeta \omega_n} = 0.028 
\label{eq::dis::tm}
\end{equation}



\subsection{Simulação do Processo}

Utilizando-se os parâmetros estimados para frequência natural e coeficiente de amortecimento, equações \ref{eq::dis::natural_frequency} \ref{eq::dis::zeta}, faz-se possível simular o processo e compará-lo com a resposta experimental. Dessa maneira, substituindo-se $\omega_n$ e $\zeta$ na equação \ref{eq::second-order-model}, temos o sistema de segunda ordem representado pela equação \ref{eq::dis::simulacao}. A simulação encontra-se expressa na figura \ref{fig::simulacao}.

\begin{equation}
G(S) = \frac{1755}{S^2 + 36.13 S + 1755}
\label{eq::dis::simulacao}
\end{equation}

\subsection{Erro em Regime Permanente}

O erro em regime permanente experimental pôde ser extraído da resposta do sistema para entradas degrau, degrau + perturbação, rampa e rampa + perturbação. O erro foi extraído das figuras \ref{fig::step_response}, \ref{fig::step_noise_response}, \ref{fig::ramp_response} e \ref{fig::ramp_noise_response} como a diferença de entre entrada e saída quando o sistema alcança o regime permanente. Também foi calculado o erro do sistema simulado, cuja função de transferência é a equação \ref{eq::dis::simulacao} e a resposta está expressa na equação \ref{fig::simulacao}. Os resultados de erros estão resumidos na tabela \ref{tab::errors}.

%% Tabela de erros em regime permanente

% Gerada a partir de: https://www.tablesgenerator.com


\begin{table}[h]
\centering
\caption{Erros em regime permanente registrados ao longo do experimento. $u(t)$ representa o tipo de entrada do sistema e $w(t)$ a perturbação adicionada. Estão representadas tantas medidas realizadas nos dados reais, como no sistema simulado.}
\label{tab::errors}
\begin{tabular}{@{}lllll@{}}
\toprule
$u(t)$ & $u(t)$ real & $u(t) + w(t)$ real   & $u(t)$ simulado & $u(t) + w(t)$  simulado \\ \midrule
Degrau   &   0.0126   &  1.4503  & $5.3160*10^{-4}$      &   $1.0011$   \\
Rampa    & 0.1344  & 1.3518 &      $0.0412$       &   $0.9593$\\ \bottomrule
\end{tabular}
\end{table}

\subsection{Questões Teóricas}
O esclarecimento das questões teóricas solicitadas em \cite{CDIN:Roteiro1} condensadas nesta subseção.

\subsubsection{"De que maneira o “turbo” afeta o experimento? ess? Valores de Km e Tm?"}



O acionamento do canal turbo proporciona uma resposta mais rápida do motor do kit impressora. Dessa forma, para o modelo da planta, pode ser interpretado como um fator que eleva a expressão de $k_m$ e, portanto, eleva o ganho do sistema como um todo e diminui a constante de tempo $T_m$. O efeito em regime permanente do sistema seria uma atenuação no erro $e_{ss}$ para a entrada rampa.

\subsubsection{"Que alteração haveria nos erros calculados (Tabela. 1) se fosse utilizado um controlador com canal integral? Qual a alteração na tipologia (para R e para W)?"}

O controlador de canal integral elevaria o tipo do sistema trabalhado nesse experimento. Dessa maneira, poderia reduzir para próximo de zero o erro em regime permanente para a entrada rampa unitária $R(S)$. Já para a entrada adicionada de perturbação, $W(S)$, o comportamento mais rápido do sistema poderia levar a instabilidades.

\subsubsection{"Considerando o controlador P utilizado neste experimento e um controlador PI, que diferença haveria na dinâmica (i.e., no tempo de pico e no tempo de subida) da resposta à perturbação?"}

A resposta do sistema seria mais rápida com a adição de um componente integrador. Dessa maneira, o sistema torna-se mais sensível às mudanças repentinas de amplitude provocadas pela perturbação. Este efeito pode levar a um tempo de subida e tempo de pico mais curtos, mas um sobresinal mais elevado.


\section{Conclusões}

O experimento realizou seu objetivo ao introduzir os estudantes a um sistema de controle real, em que realizou-se a identificação de parâmetros do kit impressora e estimou-se erros em regime permanente. A familiarização com a planta de controle introduzida neste experimento também é essencial para os experimentos futuros do laboratório de controle dinâmico.